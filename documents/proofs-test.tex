\documentclass[12pt]{article}
\usepackage{bussproofs}
\usepackage{amsmath}
\usepackage{ amssymb }
\usepackage{amsthm}
\usepackage{latexsym}
\usepackage[landscape]{geometry}
\marginparwidth=0pt
%\hoffset=-10cm
%\voffset=-5cm

\newtheorem{thm}{Theorem}[section]

% most others are numbered together with theorems
\newtheorem{cor}[thm]{Corollary}
\newtheorem{lem}[thm]{Lemma}
\newtheorem{prop}[thm]{Proposition}
\newtheorem{exmp}[thm]{Example}
%%
% just an example of what will happen if you skip the [thm]
% part -- conjectures will be numbered consecutively
\newtheorem{conj}{Conjecture}
\newtheorem{prob}{Problem}
\newtheorem{claim}{Claim}

\theoremstyle{definition}
\newtheorem{defn}{Definition}[section]
\newtheorem{property}{Property}[section]
\theoremstyle{remark}
\newtheorem{rem}{Remark}[section]
\newtheorem{notation}{Notation}

\begin{document}
\section{Local Redundancy}
We first consider an example first posed by Postan:
\begin{prooftree}
\AxiomC{$L_2$: $P(A)$}
\AxiomC{$\eta_2$: $\neg P(x)$, $\neg Q(x, B)$}
\BinaryInfC{$\neg Q(x,  B)$}
\AxiomC{$\eta_1$: $\neg P(z)$, $Q(z,y)$}
\BinaryInfC{$\neg P(z)$}
\AxiomC{$L_1$: $P(x)$}
\BinaryInfC{$\bot$}
\end{prooftree}
Which is locally redundant; see the compressed version in his document.

\section{``Example 2''}
We consider example 2, from the LU/RPI paper, modified for first order predicates in a trivial way:
\begin{prooftree}
\def\fCenter{\mbox{\ $\vdash$\ }}

\AxiomC{$\eta_1$: $\neg P(A)$}
\AxiomC{$\eta_3$: $P(A),Q(B)$}
\BinaryInfC{$\eta_4$: $Q(B)$}
\AxiomC{$\eta_2$: $P(A), R(C), \neg Q(B)$}
\BinaryInfC{$\eta_5$: $P(A), R(C)$}
\AxiomC{$\eta_1$: $\neg P(A)$}
\BinaryInfC{$\eta_6$: $R(C)$}

\AxiomC{$\eta_4$: $Q(B)$}
\AxiomC{$\eta_7$: $P(A), \neg Q(B), \neg R(C)$}
\BinaryInfC{$\eta_8$: $P(A), \neg R(C)$}
\AxiomC{$\eta_1$: $\neg P(A)$}
\BinaryInfC{$\eta_9$: $\neg R(C)$}

\BinaryInfC{$\bot$}
\end{prooftree}

\subsection{Lower Units}
Proceeds exactly the same as in the paper.\\
{\bf TODO:} show exact steps? \\

\subsection{RecyclePivots}
Again, proceeds like in the paper. 

\section{Lower Units - Research Notes}
First, I consider the proofs 1-5 that were provided by Bruno on the Skeptik dev mailing list. In order to be explicit, I outline the case of compression from proof 1 to proof 2:
\begin{itemize}
\item Lower $P(X)$ so that the terms using it were resolved against each other instead of with $P(X)$
\item Contract (trivially?); the unifier resulted in the duplicated terms
\item Resolve the contracted formula against the lowered unit, $P(X)$
\end{itemize}

The result is a trade of a resolution for a contraction, which is more compact (when we consider compactness as a count of the number of resolution rules).

In order to generalize, I think the best place to start was see under what conditions we can in fact make this contraction. It should not be required that contraction results in duplicated formulas; indeed, as long as a contraction is possible this seems to work. So in particular, I conjecture that we should lower a unit formula if and only if for all formulas which would be resolved against the unit clause of interest are pair-wise unifiable (disregarding the remainder of their premises), and unifiable with the unit. Further, the unit must be the most general form of the formula, as the following shows:

\begin{prooftree}
\def\e{\mbox{\ $\vdash$\ }}
\AxiomC{\e$P(y,x)$}
\AxiomC{$P(A,x)$ \e $Q(A),R(B)$}
\BinaryInfC{\e $Q(y), R(B)$}
\AxiomC{$Q(y)$ \e $Q(x)$}
\BinaryInfC{\e $R(B), Q(x)$}

\AxiomC{$Q(A), P(y,A)$\e}
\AxiomC{\e$P(y,x)$}
\BinaryInfC{$Q(x)$\e}

\BinaryInfC{\e$R(B)$}
\AxiomC{$R(B)$\e}

\BinaryInfC{$\bot$}
\end{prooftree}

but if we delay the resolution with $P(y,x)$ we get

\begin{prooftree}
\def\e{\mbox{\ $\vdash$\ }}
\AxiomC{$P(A,x)$\e $Q(A),R(B)$}
\AxiomC{$Q(y)$ \e$Q(x)$}
\BinaryInfC{$P(y,x)$\e$Q(x),R(B)$}

\AxiomC{$Q(A), P(y,A)$\e}
\BinaryInfC{$P(y,x), P(x,y)$\e $R(B)$}
\AxiomC{$R(B)$\e}
\BinaryInfC{$P(y,x),P(y,A)$\e}
\AxiomC{\e $P(y,x)$}
\BinaryInfC{$P(y,x)$}
\AxiomC{\e $P(y,x)$}
\BinaryInfC{$\bot$}
\end{prooftree}

and now we actually the same number of resolution rules. {\bf TODO: -no, we can still use a contraction, and reduce the proof}

The requirement for being pairwise unifiable is also seen in proof 1 and 2, but further, this is lacking the case of proof 3: $P(a)$ and $P(b)$ is not unifiable, and thus proof 5 is not actually compressed. But if $P(b)$ had been $P(B)$, then we would have been fine. It also fails in the following example:

\begin{prooftree}
\def\e{\mbox{\ $\vdash$\ }}
\AxiomC{\e $P(X)$}
\AxiomC{$P(a)$\e$Q(Y),R(Z)$}
\BinaryInfC{\e$Q(Y),R(Z)$}
\AxiomC{$R(X),P(b)$\e $S(Y)$}
\BinaryInfC{$P(b)$\e $S(Y),Q(Y)$}
\AxiomC{$S(Y), Q(Y)$\e}
\BinaryInfC{$P(b)$\e}
\AxiomC{\e $P(X)$}
\BinaryInfC{$\bot$}
\end{prooftree}

which is the 'potentially' globally reduction example from the original lower units paper.

\begin{thm}
Let $S$ be the set of premises being resolved against a unit clause $u$. Then $u$ can be lowered if, $|S|>1$ and for every distinct $\eta,\eta' \in S$, $\eta$ and $\eta'$ are unifiable.
\end{thm}

\begin{proof}
Assume that $S$ is defined as above, and is pairwise unifiable. Order the elements from the top of the proof to the bottom (and break ties left-right), so that $\eta_1$ is the top-left-most premise resolved against $u$. In particular, $\eta_1$ contains $\overline{u}_1'$, and we have that $P=\phi[\phi_1[\eta_1 \odot u] \odot  r_1]$. Consider instead $\phi[r_1 \cup \overline{u}_1']$, the proof obtained by removing the resolution $\phi_1[\eta_1 \odot u] \odot  r_1$ with just $\phi' = \eta_1 \odot r_1$ and then moving the subtree of $\phi$ to be the subtree of $\phi'$. Note that $\phi'$ contains $\overline{u}_1'$ still, and so the resulting subtree would have more occurrences of $\overline{u}_1'$. In particular, the final node in the proof is $\overline{u}_1'$ instead of $\bot$.  Since $|S|>1$, at some other point, there exists $\eta_2$ such that $\eta_2$ is also resolved against $u$ (with $\eta_2$ contains $\overline{u}_2'$. So consider  $\phi[r_2 \cup \overline{u}_2']$ and follow an argument similar to that for $\eta_1$; now the final proof node has $\overline{u}_1' \cup \overline{u}_2'$ instead of $\overline{u}_1'$. By assumption, $\overline{u}_1'$ and $\overline{u}_2'$ are pair-wise unifiable. So we can contract these terms, and then resolve against $u$, to complete the proof.
\end{proof}

\end{document}